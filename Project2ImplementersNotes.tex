

\documentclass[conference]{IEEEtran}

\usepackage{graphicx}
\graphicspath{ {images/} }

\ifCLASSINFOpdf

\else
 
\fi


\hyphenation{op-tical net-works semi-conduc-tor}


\begin{document}

\title{Project 2: Implementor's Notes}


\author{\IEEEauthorblockN{Alex Burkhart Robert McGillivray}}


\maketitle

% As a general rule, do not put math, special symbols or citations
% in the abstract
\begin{abstract}
The aim of this project is to create a working hardware description of a processor in Verilog.
An assembler specification was written to provide a starting point to writing the modules for the hardware.
As the project progressed, the assembler specification superficially changed as design decisions were made, but the core of the assembler specification remained true to what was originally written.
Before writing code, a state machine state diagram, a block diagram for control paths, and a block diagram for data paths were made.
These provided the initial top-down design used to guide how the code was developed.
Using the assembler specification and all diagrams, the modules for each piece of hardware were written.
The IDIOCC was used to generate assembly which was then run through the assembler and tested on the implementation of the microprocessor.
The processor uses 16 bit words, has 18 instructions, and will include floating point operations in the future.
\end{abstract}
% no keywords




% For peer review papers, you can put extra information on the cover
% page as needed:
% \ifCLASSOPTIONpeerreview
% \begin{center} \bfseries EDICS Category: 3-BBND \end{center}
% \fi
%
% For peerreview papers, this IEEEtran command inserts a page break and
% creates the second title. It will be ignored for other modes.
\IEEEpeerreviewmaketitle



\section{Implementor's Notes}

\subsection{Assembler Specification}
The following will outline the assembler specification for our processor design. 
Each instruction is loaded into 16 bits total with the op-code in the first 4 bits and one or two registers specified in the next 12 bits (6 bits per register).
The the op-codes for instructions and a brief explanation for each is as follows:
\begin{enumerate}
\item SYS  - Operating system call - Halts processor 
\item JZ   - Jump if zero to address in specified register
\item SZ   - Skip next instruction if value is zero
\item ADD  - Add values located in specified registers
\item AND  - Bitwise and values located in specified registers
\item ANY  - Bitwise any value located in specified register
\item OR   - Bitwise or values located in specified registers
\item SHR  - Bitshift right the value located in the specified register by 1
\item XOR  - Bitwise xor values located in specified registers
\item DUP  - Copy the value located in a register to another register
\item LD   - Load data from RAM to the register file
\item ST   - Store data from the register file in RAM
\item LI   - Load an immediate value into a reg
\item ADDF - Add 2 floating point values located in the specified registers
\item F2I  - Convert a float to an integer
\item I2F  - Convert an integer to a float
\item INVF - Compute the inverse floating point number
\item MULF - Floating point multiply
\end{enumerate}

For the purposes of Project 2, the floating point operations are not implemented.
Any floating point instruction will be encoded as SYS, halting the processor.

The complete assembler specification for this project is given below in Figure \ref{fig:assemblerSpec}.

\begin{figure}
	\caption{Assembler Specification}
	\centering
	\includegraphics[width = 9cm, height=15cm,keepaspectratio]{assemblerSpec.jpg}
	\label{fig:assemblerSpec}
\end{figure}

It can be seen that the first three instructions are encoded similarly (as in the op-codes are the same.
The reasoning for this is provided in the next section.

\subsection{State Machine Flow Chart, Data Paths, Control Paths}
The state machine for our processor is completely outlined below in Figure \ref{fig:stateMachine}.

\begin{figure}[!ht]
  \caption{State machine's state diagram.}
  \centering
    \includegraphics[width=8.5cm,keepaspectratio]{StateMachine.pdf}
  \label{fig:stateMachine}
\end{figure}

%\includegraphics{StateMachine.jpg}
The state machine which controls the processor consists of 23 states which are responsible for directing the execution of all instructions.
For one word instructions (all but LI), the controller loads the instruction into the instruction register and then executes the instruction.
If the instruction is two words long, the state machine recognizes the two-word instruction op-code from the first instruction fetch and loads the second word into the MDR. 
Since the only two word instruction is LI, there will be no conflicting states between two word instructions.
In order to fit 18 instructions into a bit field of size 4, the instructions JZ, SZ, and SYS were consolidated to a single path in the state diagram.
This was done because these three instructions perform a similar function (not to mention the assignment specified combining these specific instructions).

\subsection{Implementation Decisions}
This section will discuss the various reasons for the architecture, design, and implementation decisions.
The first major decision was to not use a separate control line for each control signal.
As the number of control signals increased, it became less and less feasible to have named wires going to and from each module.
Instead, one control line of width 64 was implemented.
The control line's control signals are as follows:

\begin{itemize}
\item controls[0] = irin;
\item controls[1] = irout;
\item controls[2] = pcin;
\item controls[3] = pcout;
\item controls[4] = mdrin;
\item controls[5] = mdrout;
\item controls[6] = marin;
\item controls[7] = marout;
\item controls[8] = yin;
\item controls[9] = yout;
\item controls[10] = zin;
\item controls[11] = zout;
\item controls[12] = regin;
\item controls[13] = regout;
\item controls[14:19] = selreg;
\item controls[20] = memread;
\item controls[21] = memwrite;
\item controls[22] = outputenable;
\item controls[23:26] = aluop;
\item controls[27] = mfc;
\item controls[28:63] = X (not used);
\end{itemize}
The controls line is of the type [0:63] instead of [63:0].
This was done for no reason other than to try it.

Each register used will output its current value or HiZ on the rising edge of every clock cycle.
Also, each register will latch values from the bus on the falling edge of every clock cycle.
This was implemented using two always blocks for both ends of a clock cycle. 
The reason for doing this was to limit the already large number of clock cycles this multi-cycle machine will take to execute a program.

A Von Neumann architecture was selected because it seemed easier to implement.
The only "difficulty" would be making sure the RAM is partitioned for instruction memory and data memory.
This "difficulty" is overcome by specifying how the RAM gets set during simulation.
AIK will output the RAM file in exactly the way it needs to be for the machine.

\subsection{Test Bench Decisions, Methods, and Coverage}
The test bench developed does several things. 
It checks the validity of each instruction (whether or not the instruction did what it was supposed to do) and it has a good coverage of testing. 
Using the coverage analyzer, the test bench developed has 100\% line coverage and 100\% finite state machine coverage. 
This means that the test bench covers each state and every path to and from each state. 
It also touches every single line of code, meaning there are no untested, nor unused, lines of code in the project.
After testing, the outputs from the register file and RAM are compared to what the actual expected values are.
For this project and test, the values all matched.
To determine whether or not the program passed, for every error a \$display will show what the error was.

Note that for running this project, the entirety of our RAM is compressed to a single 65536*16 bit wire, which increases the execution time substantially.
If the program looks like it freezes, just wait and it will eventually finish.


\subsection{Issues}
There are no known bugs or issues in the code
% You must have at least 2 lines in the paragraph with the drop letter
% (should never be an issue)

\hfill P2
 
\hfill 03/03/2016



\end{document}


